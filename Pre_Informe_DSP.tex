\documentclass{article}
\usepackage[utf8]{inputenc}
\usepackage{amsmath}
\usepackage{float}
\usepackage{graphicx}
\usepackage{float}
\usepackage[a4paper,top=2cm,bottom=2cm,left=2cm,right=2cm,marginparwidth=2cm]{geometry}

\title{Pre-Informe Trabajo final}
\author{Grupo 2}
\date{Mayo 2020 - DSP}

\begin{document}
\maketitle
\section{Marco teórico}
Con la finalidad de explicar el funcionamineto de la transformada de Wavelet se comenzara respondiendo la siguiente pregunta:
¿Porqué es importa la transformada Wavelet en el procesamiento digital de señales?.\\
Para responder esta pregunta primero definimos que es una señal estacionaria y una señal no estacionaria:
\begin{itemize}
    \item Señal estacionaria: en este tipo de señales la frecuencia no varia en el tiempo (Fig 1).
    \begin{figure}[H]
        \centering
        \includegraphics[scale=0.45]{Fig/1}
        \caption{Ejemplo de señal estacionaria. Fuente:}
        \label{fig1}
      \end{figure}
    \item Señal no estacionaria: este tipo de señales presenta variaciones en las componentes de frecuencia a lo largo del tiempo (Fig 2).
    \begin{figure}[H]
        \centering
        \includegraphics[scale=0.45]{Fig/2}
        \caption{Ejemplo de señal no estacionaria. Fuente:}
        \label{fig1}
      \end{figure}
\end{itemize}
Si queremos extraer información de este tipo de señales es inmediato pensar en el analisis en frecuencia y por lo tanto en la transformada de Fourier:
\begin{equation}
    X(\Omega) = \int_{-\infty}^{\infty} x(t)e^{-j\Omega t}dt
\end{equation}
Esta ecuación brinda una representación en el dominio de la frecuencia de una señal estacionaria, ya que $\Omega$ esta definida para todo el 
intervalo $(-\infty,\infty)$. Por lo tanto, la transformada de fourier brinda resultados óptimos cuando el contenido de frecuencia de la señal no cambia en el tiempo. Si se requiere 
analizar una señal como, por ejemplo, la de la figura 2 con la transformda de fourier, obtendriamos las componentes de frecuencia presentes en esta señal pero no tendriamos ningun tipo de 
iformacíon temporal del momento en el que cambia la frecuencia y tampoco donde se produce el spike \footnote{Irregularidad en la señal VER}. Esta información es muy importante para analizar, por ejemplo, sistemas físicos donde se quiera averiguar el tiempo en que la frecuencia de oscilación o la densidad de un material x cambie,
y utilizarla para encontrar el fenómeno que la produce.\\
AGREGAR FIGURAS QUE MUESTREN LO ANTERIOR Y TAMBIEN LO DEL SPIKE\\
Debido a este inconveniente con la transformada de Fourier, \textit{Dennis Gabor} plantea una solución que se basa en ventanas temporales. Es decir, se utiliza una ventana de longitud fija y se la desplaza 
por la señal a analizar, en cada desplazamiento se realizar la transformada de fourier a la porción de la señal encerrada en esta ventana. Esta transformada llamada STFT (Short Time Fourier Transform) lleva la información obtenida en el 
dominio del tiempo (segmento relacionado con la ventana) a la escala bidemensional de tiempo-frecuencia, dando como resultado cuándo y a que frecuencia ocurre un suceso de interés. Aunque este tipo de transformada resuelve el problema de obtener 
información temporal de los cambios de frecuencia, presenta problemas a la hora de elegir la ``mejor'' ventaja para cada señal. Por ejemplo, si las componentes de frecuencia de una señal 
estan bien separadas una de otras, se puede sacrificar resolución en frecuencia y mejorar la resolución en tiempo, esta decisión modifica la escala de la ventana a utilizar. De forma general:
\begin{itemize}
    \item Ventana estrecha: Buena resolución en tiempo pero pobre en frecuencia.
    \item Ventana ancha: Buena resolucion de la frecuencia pero pobre en el dominio temporal.
\end{itemize}
Debido a esto, el problema recae en que si la señal presenta cambios bruscos de la frecuencia, se debe elegir una ventana que actue de la mejor forma 
para todas las frecuencias presentes. Esto generaria perdidas de resolución en algunos segmentos y por lo tanto perdidas de detalles.\\
Finalmente, desarrollamos la Transformada Wavelet o tambien llamada Transformada ondeleta. Esta transformada soluciona los problemas que presenta la 
transformada de Gabor, utilizando un análisis multiresolución con ventanas de longitud variable las cuales se adaptan a la frecuencia de la señal. Es decir, mientras que con la transformada de 
Gabor teniamos una ventaja fija para el análisis, ahora es posible tener ventanas que cumplan lo siguiente:
\begin{itemize}
    \item Si se necesia mayor precisión en baja frecuencia la ventana tiene un intervalo temporal grande.
    \item Si se necesita mayor información en alta frecuencia, la ventana tendra un intervalo temporal mas pequeño.
\end{itemize}
Esto es analógo a lo presentando para la transformada de Gabor, con la principal diferencia que ahora podemos cumplir los dos requerimientos de forma simultanea. Si vemos la figura \ref{comp} observamos como la transformada de Gabor tiene ventanas de longitud fija (subfigura a).
Si se oberva la subfigura b, relacionada a la transformada Wavelet, es facil ver la diferencia en las ventanas utilizadas:
\begin{figure}[H]
    \centering
    \includegraphics[scale=0.45]{Fig/6}
    \caption{a) Esquema de las ventanas en la transformada de Gabor. b) Esquema de las ventanas en la transformada Wavelet}
    \label{comp}
\end{figure}
En la subfigura b de la figura \ref{comp} se observa en color verde la ventana que cumple el primer item. Es decir, se requiere analizar componentes de baja frecuencia y por lo tanto el intervalo temporal es grande (ventana amplia). El otro item se ve en la ventana azul, para un analisis 
en grandes frecuencias la ventana temporal sera pequeña (ventana angosta). Por otro lado, en la subifigura a) vemos la ventana en naranja correspondiente a la transformada de Gabor que no tiene en cuenta estos aspectos.\\
Estas ventanas presentadas de forma esquemática en la figura \ref{comp} tiene una forma especifica y es lo le da el nombre a esta transformada. Una \textit{Wavelet} es una ``pequeña onda'' que tiene su energia concentrada en un periodo de tiempo determinado, son de duración definida, irregulares y asimétricas, lo que les permite 
adaptarse y converger de mejor manera a la señal que se quiere analizar. Haciendo un paralelismo con la tranformada de fourier donde se utilizan senos y cosenos para representar una señal, aqui se utiliza una \textit{Wavelet Madre} y sus ``Wavelets hijas'' o ``átomos de wavelet''. Es decir, mientras que con fourier se agregan senos y cosenos para lograr 
una mejor representación, aqui se agregan Wavelets hijas. Con el fin de explicar esto planteamos la Transformada Wavelet continua:
\begin{equation}
    \label{TOC}
    T(a,b)= \frac{1}{sqrt(a)} \int_{-\infty}^{\infty} x(t) \Psi(\frac{t-b}{a}) dt
\end{equation}

Las wavelets hijas mencionadas corresponden a una traslación y/o dilatación (escala) de la Wavelet Madre ($\Psi$), segun corresponda el análisis. La escala esta representada por el coeficiente $a$ y la traslación por $b$ (en la ecuación \ref{TOC}). Si relacionamos esto con los items (poner numuero los items), si $a<1$ la información obtenida de la transformada va estar localizada en el dominio del tiempo (buena resolución temporal) y si 
$a>1$ la información esta localizada en el dominio de la frecuencia (buena resolución en frecuencia). El coeficiente $b$ simplemente desplaza temporalmente la ventana. Algunas de las Wavelets Madres mas utilizads son:
\begin{figure}[H]
    \centering
    \includegraphics[scale=0.5]{Fig/7}
    \caption{Tipos de Wavelet Madres más utilizadas. Fuente:}
    \label{WM}
\end{figure}




\end{document}